\documentclass{tufte-handout}
\usepackage{graphicx}
\usepackage{amsmath, amssymb}
\usepackage{booktabs}
\usepackage{hyperref}

\title{Design of Experiments}
\author[Yavuz Shahzad]{Yavuz Shahzad\\Winter 2025 \\
Prof. Alia Sajjad}
\date{}

% Custom Commands
\newcommand{\E}{\mathbb{E}}
\newcommand{\Var}{\operatorname{Var}}
\newcommand{\Cov}{\operatorname{Cov}}
\DeclareMathOperator{\Tr}{Tr}

\begin{document}

\maketitle

\tableofcontents

\section{Introduction}
This document provides an overview of concepts in statistically designed experiments, following the text \textit{Design of Comparative Experiments} by R.A. Bailey.

\section{Stages in a Statistically Designed Experiment}

\subsection{Consultation}
A statistician's role in an experiment ideally begins early in the design process. Effective consultation requires understanding the purpose, available resources, and timeline of the study.

\marginnote{\textbf{Example 1.1 (Ladybirds)}: A pesticide company tested a new pesticide, a standard pesticide, and a control (no treatment). Misconceptions about randomization led to flawed conclusions. (Bailey, p.2)}

\subsection{Data Collection}
A well-designed experiment ensures that data collection is structured for reliable analysis.

\begin{itemize}
  \item Each observational unit should have its own row in the dataset.
  \item Treatments should be assigned in a structured manner.
  \item Data should be retained in raw form to prevent errors.
\end{itemize}

\subsection{Data Scrutiny}
After data collection, a statistician should inspect for anomalies.\marginnote{\textbf{Example 1.3 (Leafstripe)}: A data entry mistake caused an extreme outlier, highlighting the importance of careful data review.}

\subsection{Data Analysis}
Planning data analysis prior to conducting the experiment ensures that appropriate statistical methods are used.

\section{Key Experimental Concepts}

\subsection{Replication}
Replication improves precision and generalizability. It reduces the standard error, increasing statistical power.

\subsection{False Replication}
Repeated measurements on the same experimental unit should not be treated as independent replications.

\subsection{Local Control}
Blocking is used to group similar experimental units to reduce variability, thus improving efficiency.

\section{Orthogonality and ANOVA Assumptions}

\subsection{ANOVA Assumptions}
Analysis of variance (ANOVA) requires that errors are independent, identically distributed, and have constant variance.

\begin{itemize}
  \item Normality of residuals can be checked using the Shapiro-Wilk test.
  \item A residuals vs. fitted values plot can reveal heteroscedasticity.
  \item Bartlett’s test assesses homogeneity of variances.
\end{itemize}

\subsection{Orthogonality in Experimental Design}
Orthogonal designs ensure that treatment effects can be independently estimated.

\marginnote{\textbf{Definition:} The treatment subspace $V_T$ consists of vectors that are constant within each treatment group.}

\section{Projection and Estimation in Linear Models}

\subsection{Orthogonal Projection}
If $V$ is the space of experimental units and $V_T$ is the treatment subspace, then $V$ can be decomposed as:
\[ V = V_T \oplus V_T^\perp \]
where $V_T^\perp$ is the orthogonal complement of $V_T$.

\subsection{Estimating Treatment Effects}
The best linear unbiased estimator (BLUE) for a treatment effect $\tau_i$ is given by the sample mean for that treatment:
\[ \hat{\tau}_i = \frac{1}{r_i} \sum_{T=i} Y_i \]

\subsection{Contrast and Hypothesis Testing}
A contrast is a linear combination of treatment effects that sums to zero:
\[ l_m = a_1 \tau_1 + a_2 \tau_2 + \dots + a_t \tau_t \quad \text{where } \sum a_i = 0. \]

\subsection{Projection in ANOVA}
The projection matrix for a subspace $W$ satisfies:
\[ P_W^2 = P_W, \quad P_W^T = P_W, \quad \text{and } \text{Tr}(P_W) = \dim(W). \]

\section{Conclusion}
Statistical design principles help ensure valid and efficient experiments. This document provides an introduction to key ideas such as replication, blocking, and orthogonality in experimental design.

\end{document}
